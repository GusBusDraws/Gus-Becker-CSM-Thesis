% Procedural Analyses of Image-Based, Multiscale Investigations of Materials
Methods of material investigation often produce data in the form of
images. Visual assessments can convey qualitative analyses of these
images, but expressing these initial analyses in a quantitative manner is
nontrivial. This work studies the role of image processing procedures in
extracting information for material investigations.

The first objective of this work was to explore the feasibility of
in-situ, microfocus x-radiography to monitor the composition of a binary
alloy during solidification. Features in an image are represented by
localized intensities which must be connected to physical quantities
before quantitative information can be interpreted in an image. In this
study, an image processing procedure was developed to reduce noise across
a large number of radiographs captured during directional solidification.
The final radiograph in the series, depicting the as-solidified sample,
was compared with composition data captured via energy dispersive
spectroscopy (EDS) to show radiographic intensity trends matched the
compositional trends in the EDS data.

The second objective of this work was to determine the success of image
processing procedures for identifying and tracking solid-liquid (S-L)
interfaces during in-situ, melt pool analysis of solidification
experiments. These measurements allowed for the calculation of
solidification velocity. When paired with information about the thermal
gradients in these experiments (obtainable by comparison with
simulations), this information allows for a characterization of the
structure and therefore properties of the as-solidified metal. The
detection procedures were developed for synchrotron x-radiography of
simulated additive manufacturing (AM) and dynamic transmission electron
microscopy (DTEM) of thin film rapid solidification and compared with
manual measurements. This work sought to determine whether these
procedures would be able to reduce inconsistencies due to subjective
judgment calls made during manual annotation. In the case of the AM
simulator, the results showed large deviations due to noise, but a higher
amount of the detected measurements were within the average manual
distribution. For the rapid solidification, detected results matched the
manual results closely.

The third objective of this work was to improve the segmentation of
multi-sized, irregularly-shaped, and tightly-clustered particles in a 2D
image. There are many algorithms to automatically segment features within
images, but these algorithms work best for features with uniform sizes and
shapes that have well-defined boundaries. In this study, a new
segmentation procedure is proposed which consists of three steps:
preprocessing to over-identify particles, application of a watershed
algorithm to intentionally over-segment these particles into regions, and
a custom algorithm to selectively merge these over-segmented regions based
on edge intensity between regions. The resulting merged-region results and
the results from a typical watershed segmentation are both compared with a
manual segmentation of the same image. The fit between the merged-region
results and the manual segmentation is calculated to be higher than the
fit between the typical watershed and manual segmentation.

The final objective of this work was to develop a flexible workflow for
generating 3D geometries of granular materials from microfocus x-ray
computed tomography (XCT) data. These geometries are to be used as initial
conditions in image-based physics simulations. A software package called
\textit{Segmentflow} was developed to contain the workflow functions.
\textit{Segmentflow}
is controlled by an input file which specifies the input data to be
segmented, the segmentation parameters, and the output format of the
results. The features of \textit{Segmentflow} are exhibited
by creating simulation-ready geometries from
an XCT scan of a mock high explosives system consisting of F50 silica sand 
and a Kel-F polymer binder.
The geometries are verified by analyzing the segmented particles and comparing
the results to a typical size distribution of F50 sand.
A variety of mesh postprocessing is also performed to show how
\textit{Segmentflow} can be used to control the complexity of a simulation.

