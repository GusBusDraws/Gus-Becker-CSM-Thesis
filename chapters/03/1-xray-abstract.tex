\subsection{Abstract}
High-energy, microfocus x-ray imaging, or x-radiography, is a useful tool
for in situ analysis and monitoring of materials processing. Large
fields-of-view, spatial and temporal resolutions sufficient for mesoscopic
imaging, and high-energy x-rays capable of probing metallic alloy samples
make the technique attractive for in situ solidification studies in the
laboratory. Here, we demonstrate the usefulness of high-energy, microfocus
x-radiography in the laboratory, particularly when paired with
complementary techniques. Multimodal, multiscale characterization was
performed, including x-radiographic analysis of solidifying Al-Ag and
compositional analysis of the same sample after solidification with
scanning electron microscopy (SEM) and energy dispersive x-ray
spectroscopy (EDS). The dynamics observed through x-radiography during
solidification are compared to the compositional results obtained by EDS.
The fraction solid measured in radiographs is also used in combination
with a calculation of phase diagrams (CalPhaD) Scheil solidification
simulation to reconstruct a spatiotemporal microsegregation map. The
multimodal, multiscale characterization techniques presented here
illustrate a promising pathway toward improved analyses and monitoring of
materials processing within a laboratory setting.

