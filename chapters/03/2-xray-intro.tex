\subsection{Introduction}
In situ observation is not in itself new to the study of solidification.
New methods are constantly developed and improved to capture new
information through in situ observation, as evidenced by recent reviews on
in situ solidification observation
\cite{Akamatsu2016,Shahani2020,Karagadde2021}.
Two dominant methods of producing
x-rays allow for useful in situ x-ray imaging, or x-radiography:
microfocus sources and synchrotron sources. In a microfocus x-ray source,
a stream of electrons from a high-voltage power source is focused on a
small spot on an anode. As the electrons impinge the surface of the anode,
x-rays are generated through bremsstrahlung radiation. In a synchrotron
source, x-rays are produced when high-velocity electrons in a synchrotron
facility are forced to change directions by magnetic fields. Each method
of producing x-rays has its benefits. Synchrotron x-radiography produces a
high flux of photons to capture images, enabling spatial resolutions
typically around 1 µm and temporal resolutions high enough to capture
early-stage solidification phenomena \cite{Mckeown2020}.
However, the field-of-view (FOV) in
these experiments is not as large as microfocus experiments, typically
only about 1 mm\textsuperscript{2} \cite{Mathiesen1999a}.
Additionally, these experiments require a particle
accelerator to create the radiation, so experiments need to be scheduled
and performed at a user facility. Microfocus x-ray sources are also
typically able to produce higher energy x-rays, allowing for successful
imaging of thicker or higher atomic number (Z) materials. Spatial and
temporal resolutions of microfocus systems are lower than those of
synchrotron facilities, but still competitive at up to 5 µm and 6 Hz,
respectively \cite{Rakete2011b}.

Transmittance of x-rays through a material depends on variables such as
thickness, sample geometry, and material composition. In uniformly thin
metallic samples, elemental composition plays a large role, as higher Z
atoms absorb and deflect more x-rays than lower Z atoms \cite{Heismann2003}.
The Al-Ag binary
system, while considered a model age-hardening alloy due to solid-state
precipitation behavior characterized at an atomic level \cite{Zhao2017},
is also
favorable for in situ x-radiography experiments. This is because the large
Z difference of the two elements allows for high-intensity contrasts in
captured images. This property is the motivation for using an alloy in the
Al-Ag system for the current work.

Solidification processes have been observed in situ since the 1960s when
transparent, organic compounds were observed to solidify in ways similar
to metals, exhibiting planar, cellular, and dendritic growth \cite{Jackson1965}.
In the
1970s, sealed-tube x-ray sources were first used to monitor solidification
of metals \cite{Stephenson1977}.
In the late 1980s, synchrotron x-ray topography was used to
monitor the solidification of steel using TV monitors \cite{Yamada1987},
followed by
synchrotron x-ray topography of the Al-Cu binary system \cite{Grange1994},
and finally
time-resolved x-ray imaging/x-radiography of binary alloys in the late
1990s/early 2000s \cite{Mathiesen1999a,Mathiesen2002}.
In the early 2010s, microfocus x-radiography was
suggested as a method to perform in situ solidification experiments in a
laboratory setting rather than at a synchrotron source, resulting in a
microfocus setup being brought aboard the International Space Station as a
way of performing in situ x-radiography of solidification of metals in
micro-gravity \cite{Rakete2011b},
with similar experiments performed on sounding rockets \cite{Nguyen2013}
and parabolic flights \cite{Murphy2014}.

Synchrotron x-radiography has been used to map x-ray intensity to
composition through comparison with analysis performed with an electron
probe microanalyzer (EPMA) \cite{Husseini2008}.
Compositions have also been applied to
x-radiography intensity, using calibration experiments involving phases of
known composition to create a mapping function from radiograph intensities
to composition
\cite{Husseini2008,Griesche2010,Han2017,Ruvalcaba2007,Bogno2011,Bogno2013,Becker2016a}.
This type of method was used, for
instance, to determine concentrations of solidifying Al-Ge from
x-radiography intensities to compare with dendritic needle network
simulations of the same system \cite{Becker2020,Tourret2016}.
These works provide a promising
method for relating x-radiography intensity to composition when a
calibration experiment is involved. When calibration experiments cannot be
performed, it is of interest to determine whether x-radiography pixel
intensities can still be used to infer in situ composition values during
processing of materials. This current work explores the ability of pairing
microfocus x-radiography with post-solidification compositional analysis
similar to the aforementioned EPMA method, but more suitable for a
laboratory setting.

On its own, x-radiography reveals relative variations affecting x-ray
transmittance through a sample (e.g., Z differences), so it cannot be used
to quantify composition gradients in a solidifying alloy, but certain
types of microscopy are able to measure composition post-solidification.
Scanning electron microscopy (SEM) is a versatile technique in which a
focused beam of electrons is rastered across a sample surface to produce a
variety of signals collected for imaging and microanalysis on the
sub-micron scale. Two common signal types are backscattered electrons
(BSE) and characteristic x-rays. Backscattered electrons are used to image
microstructures. Contrast in these images is strongly influenced by
compositional variations across the sample surface, because the BSE signal
intensity increases with Z \cite{Reuter2003}.
For this reason, BSE imaging is commonly
paired with compositional mapping performed through energy dispersive
spectroscopy (EDS) of characteristic x-rays.

In this article, we demonstrate a laboratory-based, multimodal approach to
the multiscale characterization of metals during solidification, using the
Al-Ag system as an example. This technique bridges the mesoscopic length
scale of high-energy, microfocus x-radiography with the microscopic length
scale of EDS. By correlating data from each method of analysis, one can
obtain a greater amount of information than could be determined from
either method of analysis on its own. Scheil solidification calculations
are performed to further contextualize the results and to provide a
methodology to link phase fractions and compositions obtained during
solidification. We also discuss some promising pathways linking
simulations and experiments to further expand the breadth of data
extracted from both in situ and post-solidification analyses.

