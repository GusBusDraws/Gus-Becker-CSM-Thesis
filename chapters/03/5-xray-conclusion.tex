\subsection{Conclusion}
This work demonstrates a monitoring technique combining the mesoscopic,
volume-probing investigation of high-energy, microfocus x-radiography and
the microscopic, compositional analysis of SEM EDS, supported by Scheil
solidification calculations. In situ x-radiography can provide relative
composition and solid fraction evolutions by pixel intensity changes
during solidification, while post-solidification EDS can provide more
quantitative compositional information, but only after processing of the
material and with localized surface analysis. The pairing of the data
received through volume-probing x-radiography with compositional surface
analysis captured by EDS creates a multimodal montage of data that is
greater than the sum of its parts. The combination of these concurrent
datasets with modeling data, here illustrated using simple CalPhaD-based
Scheil calculations, could reveal nuances beyond those directly accessible
through either the imaging, compositional mapping, or simulations taken
separately. Further development and calibration of the techniques
illustrated here will allow for in situ compositional monitoring
capabilities of dynamic processing of metals within a laboratory setting.
Potential exists for a completely non-destructive, in situ monitoring
technique available in the laboratory, with which a sufficient number of
samples could be analyzed by x-radiography and EDS, supported by modeling
data and perhaps calibration experiments, to provide a comprehensive way
to capture important information about solidification dynamics.
Furthermore, a database could be created to link radiograph intensities
with EDS-derived compositions for enough relevant conditions (i.e., sample
chemistries and geometries, processing conditions, etc.), combined with
modern, fast-acting real-time post-processing (e.g., based on
machine-learning algorithms), such that compositions could be inferred in
situ with reasonable accuracy.

\subsection*{Acknowledgements}
The microfocus x-radiography was supported by
the US Department of Energy, Office of Science,
Basic Energy Sciences under award no. DESC001606.
The SEM EDS and preparation of this
manuscript were supported by the US Department
of Energy, Office of Science, Basic Energy Sciences
under award no. DE-SC0020870.
Los Alamos National Laboratory is operated by Triad National
Security, LLC, for the National Nuclear Security
Administration of US DOE (Contract No. 89233218CNA000001).
The Tescan dual-beam FIB was funded through the support of the National
Science Foundation (DMR-1828454). Many of the
figures in this text were generated using Python \cite{python},
Matplotlib \cite{matplotlib}, and the open-source vector editing
software InkScape \cite{inkscape}.

