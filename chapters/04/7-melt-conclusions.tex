\subsection{Conclusions}
Methods for calculating solidification velocities are important
for accurately predicting microstructures in AM-like processes
and linking those microstructures to processing parameters.
Characterization tools like synchrotron
x-radiography and DTEM enable high spatial and temporal resolution in-situ
monitoring of solidification, but the large amounts of data these
techniques yield require difficult manual analysis to retrieve
quantitative information like solidification velocities. Tracking and
annotating image data manually introduces human bias and error that can be
inconsistent across measurements and users. Reasons for these
inconsistencies include subjective judgement calls, personal preferences,
miscommunications, and user fatigue. These sources of error in
measurements can be difficult, and in some cases nearly impossible, to
quantify. The automated detection techniques presented in this work
explored the possibility of a consistent method of analyzing image data
from solidification experiments, while also reducing the effect of bias on
the collected data. Two procedures were presented in this work for
automating the detection of S-L interfaces and calculating solidification
velocities, and each procedure was applied to three separate experiments.
The first procedure determined S-L interface positions in synchrotron
x-radiography images from simulated AM experiments. The second procedure
automated the detection of S-L interfaces for frames in three
separate DTEM images depicting thin film rapid solidification.
Each procedure was similar but each took a slightly different approach to
interface detection.

The simulated AM procedure performed adequately as compared to manual
measurements.
The detection procedure returned drastically different mean velocities from
the manual measurements for the 104 W and 156 W experiments, however the median
detected and manual velocities are similar, suggesting that the differing
means are because of outliers in the velocity, which is supported by the
significantly higher detected deviation from the manual mean velocity
compared with the manual deviation from the manual mean velocity.
In the 208 W experiment, the detected mean velocity is the same as the
manual velocity, but the median velocities differed. Similar to the 104 W
and 156 W experiments, the detected deviation from the manual mean is much
higher than the manual deviation. This suggests that the 208 W experiment
has a similar problem with the outliers, but that the outlying detected
velocities are more evenly distributed both above and below the manual
mean velocity.

The rapid solidification procedure performed well as compared to the manual
measurements. For the first and second experiments, the mean and median values
are both similar with low deviations from the manual mean velocity. In the
third experiment, the detected mean velocity differs more than either of the
previous two experiments, but the median values are still similar. The
detected deviation from the manual mean is also higher in this experiment.
Upon closer analysis, the detected velocity has the highest deviation from
the manual mean velocity towards the end of the experiment. This is
significant because the third experiment fully solidified
before the other two experiments. The melt pool does not even show up
in the final frame. This suggests the procedure does best when
the melt pool is larger and more distinguishable from noise.

The simulated AM procedure did not include any kind of a fit optimization step.
The rapid solidification procedure, which did optimize the fit of the detected
melt pool, was more successful in matching the manually measured velocities.
If a similar optimization step could be included in the simulated AM procedure,
there may be an increase in detection accuracy.
In addition, the fitting of the interfaces to parameterized shapes would allow
the solidification velocity to be calculated at multiple points along the
interfaces. This is not currently possible since the procedure
measures the location of the interface based on the bottom of the interface,
thus only allowing for the vertical solidification velocity
to be calculated.

While this work doesn't claim that these data extraction and analysis methods
remove human biases altogether (the procedures were written by a human, after
all), a procedure deterministically run by a computer
makes the method of extracting data more consistent. The nature of the
procedure development may also prompt researchers to more
carefully consider biases introduced through their methods.
Each step in the procedure must be actively considered to be incorporated,
whereas unconscious passivity is possible in manual actions
performed in a repetitive manner.
The less laborious effect of automated procedures may also benefit researchers
by allowing them to more quickly and easily test the effects
of small changes to data extraction and analysis, as opposed to the significant
time and effort to reproduce changes to manual routines. This reduced barrier
to iteration could encourage researchers to test a larger number of iterations
data extraction methods, potentially even resulting in more refined results.

\subsection*{Acknowledgements}
A special thanks to the beamline scientists at the Advanced Photon Source
that make themselves available at odd hours to keep experiments at the
facility running smoothly and to the students of the Center for Advanced
Non-Ferrous Structural Alloys that agreed to work all the other odd hours to
run the experiments and collect the simulated AM data.

