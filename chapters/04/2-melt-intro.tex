% -------------------------------------------------------------------------
\subsection{introduction}
% -------------------------------------------------------------------------
Metal additive manufacturing (AM) encompasses a promising collection of
manufacturing techniques in which metallic parts are created layer by
layer. These techniques can create parts with complex geometries that are
not achievable with more traditional, subtractive techniques
\cite{Chu2008}. However, these techniques are not without their own
challenges. AM-built parts often produce columnar dendrites during
solidification. These anisotropic microstructures can lead to hot tearing
\cite{Kaufmann2016,Martin2017,Gu2019}.
% and anisotropic material properties?
While there has been some success
leveraging anisotropy to intentionally localize properties in AM-built
parts \cite{Dehoff2015,Plotkowski2021}, isotropic microstructures are
usually preferred to reduce the tendency for cracks to form. This requires
the growth of equiaxed grains as opposed to columnar grains. Many methods
have been successful in encouraging equiaxed grain growth, including
alteration of alloy composition \cite{Spittle2013,Bermingham2019,Zhu2019},
addition of grain nucleating nanoparticles \cite{Martin2017,Gu2019}, changing
build height \cite{Qiu2015}, changing feedstock rate
\cite{Wu2004,Qiu2015}, and changing laser/electron beam processing
parameters like power \cite{Qiu2015,Balla2016,Kurzynowski2018}, scan speed
\cite{Wu2004,Qiu2015,Balla2016}, scan strategy
\cite{Dehoff2015,Kurzynowski2018}, and beam shape \cite{Roehling2020}.
A series of models have enabled studies of the effects of beam
processing parameters on melt pool dynamics and resulting microstructures.
The first of such models was developed by Hunt to analytically describe
the growth of equiaxed grains ahead of the columnar solid-liquid (S-L)
interface, shedding light on the columnar-to-equiaxed transition (CET)
\cite{Hunt1984}. This original CET model was developed for casting
applications, but the Kurz-Giovanola-Trivedi model describing rapid
solidification in the growth of columnar dendrites \cite{kgt1986} enabled
Gäumann et al. to extend Hunt's model to rapid solidification
\cite{Gaumann1997,Gaumann2001}. The latter work of Gäumann et al.
provided a simplified relationship between temperature gradient (G),
solidification velocity (V), and a material constant (K), which enables
predictions for when an alloy system undergoes equiaxed versus columnar growth.
Using this relationship, studies have predicted and verified
microstructures for varying processing parameters by comparing V, as
calculated from experiments, and G, as estimated with heat transfer
simulations in G-V maps
\cite{kgt1986,Kurz2001,Kurz2001b,Gaumann2001,Kobryn2003,Dehoff2015,Saville2021,Roehling2020,}.
These microstructure maps overlay experimental data with regions corresponding
to columnar, equiaxed, and mixed microstructures. Microstructure
maps can also be express process parameters directly, rather than G
and V, to determine combinations of processing parameters likely to
produce optimal microstructures \cite{Liang2016,Liang2017}.

With the opportunity to view melt pool evolution and development of
microstructures under AM-like conditions in situ comes the task of
tracking image features across the duration of solidification.
This feature tracking is often performed manually with some kind of
image viewing and annotation software like \textit{ImageJ} \cite{imagej}.
Manual tracking has been used to track droplet spatter
\cite{Leung2018nat,Leung2018am}, melt pool length/depth and overall area
shrinkage \cite{Leung2018am}, pores \cite{Leung2018nat,Martin2019}, vapor
depression/keyhole depth \cite{Martin2019,Cunningham2019}, powder motion
and spattering \cite{Guo2018}, melt flow within the melt pool via tracing
particles \cite{Hojjatzadeh2019}, and melt pool volume \cite{Guo2019}.

Most studies in literature do not explore automated methods of analyzing
melt pool features, although a procedural method for identifying interfaces
of melt pools has received a brief explanation in some studies
\cite{Zhao2017,Wolff2019}. In these studies, the interfaces
are identified on a row-by-row basis, based on local peaks in the second
derivative of the radiographs. This methodology is provided in the
supplementary materials, however, and no further explanation is given into
how the peaks of the derivatives corresponding to the interfaces are
selected from the other peaks present.

The goal of this work was to investigate the possibility of automating
analyses necessary to calculate solidification velocities for the
prediction of microstructure based on processing parameters. The
successful execution of this type of automation would remove some inaccuracies
related to human error and inconsistent subjective judgements across
different researchers, while also improving analysis efficiency. Automatic
analysis procedures are developed for two solidification experiments: an
AM simulator experiment and a rapid solidification experiment. The AM
simulator was developed at section 32-ID-B of the Advanced Photon Source
(APS) synchrotron facility at Argonne National Laboratory \cite{Zhao2017}
to simulate laser powder bed fusion (LPBF): an AM
technique in which material is fused together, one layer at a time, using
a high-energy laser \cite{King2015}. That said, the AM simulator may also be
used more generally to observe laser-substrate interactions (i.e., a
substrate without a powder layer).
Hard x-rays and high-speed x-ray detectors are used
to image the melting and solidification of Ni-1.9 Mo-6.6 Al (wt.\%)
single crystals in situ, revealing the melt pool such
that the S-L interfaces can be identified and tracked. The resulting location
information of the S-L interfaces are used to calculate solidification
velocities under LPBF-like conditions. The rapid solidification
experiments were performed by melting thin films
(approximately 100 nm in thickness)
of Al-3 wt.\% Si on an amorphous silicon nitride substrate and
monitoring the solidification using a dynamic transmission electron
microscope (DTEM) at Lawrence Livermore National Laboratory. DTEM has
proven to be useful in many rapid solidification studies
\cite{LaGrange2006,LaGrange2008,Kim2008,Campbell2010,Kulovits2011,LaGrange2012,Santala2013,McKeown2014,Zweiacker2015,LaGrange2015,Roehling2017,Ji2023}.
% Split this into many solidification experiments including x, y, z?
For each analysis procedure performed here, the automated measurements
are compared to manual measurements to assess the performance of the procedures.

