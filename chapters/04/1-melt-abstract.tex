% \section{Abstract}
In this chapter, procedures for detecting metallic melt pools are
presented to automate the
process of obtaining real-time solidification velocities to enable
comparisons to process modeling and microstructural outcomes predicted by
solidification theory and modeling. Procedures are developed to analyze
two types of solidification experiments: simulated metal
additive manufacturing (AM) of Ni-1.9 Mo-6.6 Al (wt.\%)
single crystals captured with x-radiography at the Advanced Photon Source
(APS) synchrotron facility at Argonne National Laboratory (ANL) and rapid
solidification of Al-3 wt.\% Si thin films captured with dynamic transmission
electron microscopy (DTEM) at Lawrence Livermore National Laboratory
(LLNL). Each procedure differs due to the nature of the experiment, but
each utilizes common Python libraries including \textit{NumPy},
\textit{imageio}, \textit{scikit-image}, and \textit{napari},
to perform steps including denoising, pseudo-flat-field intensity
correction, segmentation, and optimization of ellipse fitting.
The procedures are applied to three experiments of each type and the
detected melt pool evolution is compared with
manual measurements to show that the procedures are reasonably accurate.
The AM simulator procedure is more prone to inaccuracies due to noise,
and therefore less reliable than the rapid solidification procedure, which
is more robust, partly due to a fit optimization step.

