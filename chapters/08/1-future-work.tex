% Chapter 7: Recommendations for Future Work
The work presented in this thesis answered questions that were sought at
the beginning of the project. In the process, however, more questions
were discovered. These discoveries provide opportunities
to compound the results presented in this thesis.
This chapter addresses these opportunities by providing 
recommendations for future work. 

\subsection{Optimization of AM Simulator Detection}
% -------------------------------------------------------------------------
In Chapter \ref{ch/melt}, two procedures were presented detect the
solid-liquid interface in a solidifying metal sample. The procedure
developed first was for AM simulator experiments and was designed to
detect semi-elliptical melt pools. The second procedure,
developed after the first, was designed to detect elliptical melt pools in
thin film, rapid solidification experiments.
The procedures were similar, but the AM simulator procedure detected the
bounding box of the semi-elliptical melt pools whereas the rapid
solidification experiment optimized the fit of an ellipse to the
elliptical melt pool. The rapid solidification procedure performed better
than the AM simulator procedure. The incorporation of fit optimization may
not have been the only reason for the improved performance, but
incorporating some kind of an optimization step into the AM simulator
would be a good place to start improving the procedure.

The optimization included in the rapid solidification procedure optimized
the fit of an ellipse to the melt pool. To bring optimization to the AM
simulator procedure, a shape could similarly be fit to the data. However,
the fit of the ellipse was optimized by minimizing a cost function that
incorporated the mismatch of the ellipse with a binary image
representation of the experiment. A cost function to be optimized to
improve the fit of the AM simulator could include many different
parameters besides fit mismatch. Other parameters could
include detected melt pool aspect ratio, difference in size
with the melt pool of the previous frame, difference in location of
with previous melt pool, location relative to the center of the image,
or location relative to the top of the sample.
Exploring the optimization using some of these parameters would make for
an interesting study that is likely to achieve promising results based
on the number of possibilities alone.

\subsection{Extension of Edge Strength Corrected Segmentation}
% -------------------------------------------------------------------------
In Chapter \ref{ch/seg}, a method was presented to correct the
segmentation of irregularly-shaped, multi-sized, and tightly-clustered
particles. The implementation of this algorithm was successful,
however the presented routine was only applied in two dimensions.
The image on which this routine was tested is itself from a
3D dataset. This displays the biggest opportunity to be
pursued following this work: adapting this routine to work with
3D data. The watershed segmentation and Delaunay
triangulation algorithms leveraged for this procedure can both function
in three dimensions, so the main focus would need to adapt the region
merging algorithm to 3D. A potential obstacle is the increase in the number
of neighbors for marker will in the adaption to 3D. This may require
optimization of the merging algorithm, depending on the size of the
datasets to which the procedure is applied.

Another path for continued study related to this method is further
development of the criteria defining detected edges.
The method as it stands has two conditions for detecting edges: a simple
threshold dependent on the maximum intensity of the edge-amplified
image and the location of any detected edges along the line connecting
two markers/regions. Perhaps by applying more thorough signal
processing techniques, a more robust criteria could be developed that
wouldn't require fine-tuning on an image-to-image basis that is likely
necessary with the current algorithms. A study on either or both of these
extensions to this method would be interesting and useful, especially for
the application of image-based simulations as discussed in Chapter \ref{ch/sf}.

\subsection{Additional Segmentation Functionality for Segmentflow}
% -------------------------------------------------------------------------
Chapter \ref{ch/sf} presented \textit{Segmentflow}, a Python package for
developing and/or executing segmentation workflows to create geometry
that could be used as initial conditions in a physics simulation.
\textit{Segmentflow} is a flexible software library that has been steadily
increasing the more it is used and the more diverse its user base has
become. There are many additional features that could be added, but some of
the most scientifically interesting include additional methods for
determining particle size distributions and the ability to merge
segmentation results. In Chapter \ref{ch/sf}, size distributions were
calculated for the segmented particles by binning particles by diameter as
determined by either assuming the particle was a sphere or using the maximum
bounds of a particle. The sphere method is obviously limited because
most particles are not actually spheres. For very blocky
particles, this may be an underestimate of size, but for most particles,
this will be a overestimate of size since the particles may have volume
distributed preferentially along one or two axes rather than all three
equally. The limitations of the bounding box method, which determines the
size based on the aspect ratio of the smallest box a particle could fit in
does a better job of not overestimating size based on volume distribution,
however, there are further limitations related to how a particle is oriented
within the box. The box is constrained to the axes of the voxels, so if an
oblong particle is oriented significantly askew to any of these axes, the
bounding box will be much larger than the particle. A better method for
approximating the size distribution of segmented particles would be to fit
a tightly fitting ellipsoid around each particle, however this is not
currently possible in \textit{Segmentflow}.

As discussed in Chapter \ref{ch/seg}, segmenting irregular
particles is difficult because if results are a mix of over- and
under-segmented, the segmentation cannot be improved in both directions,
which was the motivation for the Chapter \ref{ch/seg} in the first place.
Perhaps rather than trying to correct results from watershed segmentation,
methods of segmentation could be investigated. Many of the
incorrect results that can result from watershed segmentation have complex,
unnatural-looking forms. Rather than try to correct these forms after
segmentation, a new segmentation method might be able to avoid these forms
in the first place. One potential path for
this kind of a study would be to investigate a method to
incorporate minimization of surface energy into a segmentation algorithm.

