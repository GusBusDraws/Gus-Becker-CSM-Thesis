% 5.1 Introduction
\subsection{Introduction}
\textit{Segmentflow} is a Python package for creating segmentation
workflows which
convert 3D image data to a format usable in image-based simulations.
This software has been designed to be flexible enough
to be used for a variety of applications and data types. Part of what
makes \textit{Segmentflow} flexible is the ability to tune the segmentation
process and the format of the resulting outputs.

\textit{Segmentflow} was designed
to extract the geometries of granular particles embedded
in a polymeric matrix for image-based simulations.
Specifically, the behavior of these particles would be simulated in
quasi-static, unconfined compression by performing a
direct numerical simulation (DNS) on the real geometries
extracted with \textit{Segmentflow}.
These samples are imaged using x-ray computed tomography (XCT).
For the granular particles, the goal of the segmentation is to separate
each particle from the other particles and the polymeric binder,
then to convert the grayscale intensity voxel data to
a labeled voxel format and a collection of surface meshes
(one surface mesh for each particle).
Each of these output types would be
used as inputs for a different type of simulation software.

\textit{Segmentflow} ties together the functionality from many powerful
Python libraries. To improve make interfacing with \textit{Segmentflow}
more user-friendly, YAML files are used to pass input parameters to
\textit{Segmentflow} functions using the package \textit{PyYAML} \cite{pyyaml}.
To load and save images, the package \textit{imageio} \cite{imageio}.
Image data and manipulations are performed using \textit{NumPy} arrays
\cite{numpy}. Viewing images and plotting data and information about these
images is accomplished with \textit{Matplotlib} \cite{matplotlib}.
Many useful image processing algorithms implemented in
\textit{scikit-image} are utilized \cite{skimage}. To save surface meshes
in the STL file format, the package \textit{numpy-stl} is used \cite{numpystl}.
Finally, image postprocessing methods as well as 3D visualization rely on
the powerful 3D data processing library \textit{Open3D} \cite{Zhou2018}.

There are two types of segmentation, and the distinction between these types
is important for \textit{Segmentflow} because many segmentation workflows
depend on both types.
The first type of segmentation, semantic segmentation, is the process of
separating, or segmenting, features in an image by type or class.
Semantic segmentation on its own can be useful to determine information
like volume fractions, but often a semantic segmentation is the first step
for a more involved analysis.
There are different methods for performing semantic segmentation,
often by setting a series of threshold values in an image
and classifying features between those values.
This can be done manually or procedurally. A common algorithm for
procedurally determining threshold values is
known as multi-Otsu thresholding \cite{Otsu1979}.
The second type of segmentation is instance segmentation.
In instance segmentation, a region/class of an image,
is segmented into individual feature
instances, or segmented particles, of that class.
Often instance segmentation operates on a binary
representation of a class previously determined via semantic segmentation.
These segmented particles will be identified via labeled
voxels in which each voxel corresponding to a unique particle will be
labeled with a unique integer.

A common type of instance segmentation
is watershed segmentation.
Watershed segmentation can be
thought of as a simulated flooding process applied to an image
\cite{Beucher1979,Beucher1992,Soille1990viscomm,Soille1990sigproc,Vincent1991}.
As a watershed algorithm is applied to an image, the intensities of the
image act as elevations of a topographic surface. Simulated ``water'' can be
filled from the lowest elevations first, equal across the full image,
or the water can be ``poured'' from specifically locations called markers
\cite{Moga1998,Parvati2008}. As the water fills the surface,
separate ``catchment basins'' regions that started filling in different
places begin to near one another. Any time two catchment basins come into
contact, a ``dam'' is created. The algorithm finished when the image a has
been completely filled, or reaches a specified elevation/intensity. The
resulting dams denote the segmented regions.
It is common for watershed segmentation to be
applied to a distance transformation of a binary image, which in the case
of the Euclidean distance transformation (EDT), assigns a new
value to each foreground pixel according to the distance to the nearest
background pixel \cite{Danielsson1980}. This creates an image that is more
logically akin to an elevation map, as the distance map has a natural
gradient. The distance transform of a few black spots on a white image
would appear as a mountainous island, with the peak of the island at the
farthest point from the shore of the metaphoric white sea. If the distance map
is inverted, a gradual depression is formed that can be filled by the
algorithmic flooding using information about the morphology of the
foreground region in an image to segment the features.

