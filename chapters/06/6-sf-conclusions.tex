% 6.6 Conclusions
\subsection{Conclusions}
This work presents \textit{Segmentflow}, a segmentation workflow tool
that can be used to generate geometries for use in physics simulations.
A workflow is demonstrated that takes a CT scan of a mock high explosives
system consisting of F50 sand grains coated in a polymeric binder.
The CT scan was subject to a preprocessing procedure to increase the
contrast of the image, at which point it was semantically segmented
into regions labeled as one of three classes: void, binder, or sand grain.
The regions corresponding to the sand grain class were subject to an
instance segmentation procedure with the goal of segmenting and labeling
voxels identified as individual sand grains. These segmented particles are
dependent on the markers used to seed the segmentation algorithm.
A variety of markers were selected by specifying a range of minimum
distances separating the points. Each set of markers produced a different
segmentation. The segmented particles resulting from each separate
segmentation were analyzed to assess the segmentations.

Segmented particles resulting from five segmentations were analyzed.
The size distribution of the segmented particles
for each segmentation was compared with the typical F50 sand size
distribution to determine which segmentation yielded the most
accurate segmentation. Two methods were used to determine the size
distribution of the segmented particles from each segmentation. The first
method used the volume of each segmented particle and calculated the diameter
of the sphere of equivalent volume. This diameter was used to bin each
particle between the mesh sizes defined by the typical F50 distribution.
The second method used the aspect ratio of the bounding box of each particle
and used the maximum length of the minimum cross section of the box to bin
the particles. The sum of accumulated error was calculated for each method
and the total accumulated error, summing the error from each method,
was calculated for each of the five segmentations. The segmentation that
was seeded by markers with a 6 pixel (83.04 µm) minimum peak distance was
determined to have the lowest total accumulated error. A slice corresponding
to the same location of the sample volume was taken from each of the
three segmentations with lowest error and compared to provide a visual
example of the quality of the segmentations.

The segmented particles resulting from the segmentation with lowest error
was used to demonstrate the surface meshing capabilities of
\textit{Segmentflow}. The surface meshes were visualized together to show
the complete geometry of the sample. Postprocessing capabilities were also
demonstrated by comparing the unaltered surface mesh with smoothed and
simplified representations. The particle was shown to maintain its
general shape even when the number of triangles making up the surface
was reduced by more than two orders of magnitude. This shows an important
ability of \textit{Segmentflow} to control the complexity of geometries
and therefore the simulations for which these geometries will be used.

