% Implementing edge strength corrections to improve segmentation of
% irregularly-shaped, multi-sized, and clustered particles in
% two-dimensional images
\subsection{Introduction}
Image segmentation often plays a crucial role in digital image processing
workflows. Segmentation enables the extraction of quantitative information
from images that is only qualitatively available through visually analysis
alone. Segmentation can be split into two categories: semantic segmentation
and instance segmentation. In semantic segmentation, an image is separated,
or segmented, into classes for analysis, often the regions of interest
(the foreground) and the background.
The result of this type of segmentation is a binary
image, so this process can be referred to as binarization. A simple way to
binarize an image is by selecting a threshold value such that pixels with an
intensity above this value are categorized as one of the binary labels,
while the pixels with an intensity below the thresholding value are
categorized as the second label. The threshold value is often selected
manually, but an algorithm can also be used to calculate this value. A
useful and well-known thresholding algorithm is Otsu's method \cite{Otsu1979},
which computes an
optimal threshold value by analyzing the image histogram. The histogram is
split into classes at different values, and a ``goodness'' factor, defined
by considering within-class and between-class variances, is optimized to
determine the threshold value.

The second type of segmentation, instance segmentation, is used to separate
specific instances of a class within an image.
Instance segmentation can
separate features within the foreground from one another to enable analyses
including feature locations, sizes, and shapes.
In some cases, it may be possible to separate multiple objects based
on gray value alone by setting multiple threshold values,
however instance segmentation can also be performed by analyzing the
morphology of a semantic segmentation.
One such method involves
morphological watersheds, and is typically referred to as watershed segmentation
\cite{Beucher1979,Serra1988,Soille1990sigproc,Soille1990viscomm,Beucher1992}.
Watershed
segmentation algorithms take their name from the analogy of ``flooding'' an
image by interpreting each intensity value as a height, as if the image is
describing a topographic surface. When a flooding operation is performed
algorithmically, the local minima of the image act as ``catchment basins''
which fill until a flood reaches a neighboring flood, at which point a
``dam'' is created to prevent the floods from merging. After this process is
completed, the lines representing the dams remain, thus segmenting the
image. As most images wouldn't appear to represent a logical
three-dimensional surface, a transformation of some kind is often applied
before watershed segmentation. This can be done using the Euclidean
distance transformation, which maps each foreground pixel of a binary
image to an intensity representing the distance to the nearest background
pixel \cite{Danielsson1980}. Distance transformations are usually inverted
before applying a
watershed segmentation algorithm so the pixels farthest away from the
background (local maxima) will be converted to local minima
that will be filled first by the flooding simulation.

Watershed segmentation algorithms can operate by providing a list of
markers which seed the catchment basins used in the algorithm
\cite{Moga1998,Parvati2008}. These
markers provide locations to begin algorithmic flooding at the same time,
regardless of the pixel intensities (``heights'') of these locations.
Markers can thus be
thought of as ``pour points.'' This can be useful to control
over-segmentation (more segmented regions than expected) and
under-segmentation (fewer segmented regions than expected) because in
marker-defined watershed segmentation, the number of markers determines
the number of catchment basins and therefore the number of final segmented
regions. The limitations of segmenting an image via watershed algorithm
are often related to the method employed to define the markers. A
common method of generating markers is to calculate the minima of the
inverse distance transformation. However, this can lead to spurious local
minima for irregularly shaped and/or clustered particles, resulting in
over-segmentation \cite{Sun2017}. In these cases, rather than calculate
all local minima, \textit{h}-minima can be calculated such that each minima
is a value \textit{h} above any surrounding minima \cite{Cheng2009,Jung2010}.

A downside to using \textit{h}-minima for marker generation is that this can
suppress legitimate minima corresponding to smaller particles when a variety
of particle sizes are present. This has been addressed with an adaptive
\textit{h}-minima method for selecting watershed markers \cite{Burgmann2022}.
This method uses dynamic \textit{h} values determined by setting a lower and
upper \textit{h} value for the intensity range of the image. For an inverse
distance transform, the \textit{h}
value range is set such that minima with lower intensities (i.e., points
farther from edges) have a higher \textit{h} value than minima with higher
intensities (i.e., points closer to edges). This translates to allowing
minima to be selected that are closer together around small objects in the
image but farther apart around larger objects. Another method is to remove
minima within a set radius determined by the comparison of intensities
between each minima and the surrounding minima \cite{Sun2017}.
This is based on the
assumption that spurious minima are caused by irregularly-shaped or
overlapping objects. Rather than suppress markers before seeding a
watershed algorithm, another method addresses over-segmentation by
detecting strong edges using the Canny edge filter and merging segmented
regions if the shared boundary does not exist on an edge, as long as the
edge is strongly defined \cite{Canny1986,Kim2003}.

