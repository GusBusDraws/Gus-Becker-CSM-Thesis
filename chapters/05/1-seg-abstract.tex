% \section{Abstract}
Watershed segmentation algorithms are useful tools
to separate features in an image from one another, enabling
more targeted analyses. However, these algorithms
often do not achieve expected results for multi-sized or
irregularly-shaped particles, or particles clustered tightly together without
strongly defined edges along contact surfaces.
This chapter presents an image processing procedure
implemented in Python (with
libraries including \textit{scikit-image}, \textit{imageio}, \textit{NumPy},
and \textit{SciPy}) to improve the results of a
watershed segmentation algorithm. The proposed procedure involves application
of a preprocessing routine and custom algorithm which utilizes Delaunay
triangulation to merge segmented regions based on
edge strength between the regions.
The resulting merged-region segmentation results are calculated to more
closely match a manual segmentation than a typical watershed segmentation
on its own. The procedure is tested on a 2D radiograph sliced from a 3D
x-ray computed tomography (XCT) dataset depicting irregularly-shaped and
multi-sized sand grains that are tightly clustered with a polymer binder.

