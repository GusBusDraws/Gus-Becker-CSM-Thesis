\subsection{Conclusions}
This work presented a procedure to improve the segmentation of
irregularly-shaped, multi-sized, and tightly-clustered particles
using a method that merges regions based on edge strength.
A 2D image taken from a 3D XCT scan of polymer-bonded sand grains was
used as an example image. The procedure preprocesses the image to
create a binary image separating the sand grains from the background and
generates a set of markers to intentionally over-segment the particles.
The majority of the procedure lies in a custom algorithm which utilizes
Delaunay triangulation to merge neighboring regions that are not found to
be separated by an edge.

The merged-region segmentation appears to be an improvement over a typical
watershed segmentation, however a manual segmentation was also performed
and used as a baseline to quantitatively assess the improvement.
The fit between the manual segmentation and merged-region segmentation is
calculated and compared to the fit between the same manual segmentation
and the typical watershed segmentation.
The merged-region segmentation was calculated to be an 89.02\% fit with
the manual segmentation. This is a 6.93\% improvement over the typical
watershed segmentation, which was calculated to be an 82.09\% fit with the
same manual segmentation.

