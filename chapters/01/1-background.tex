% \section{Background}
% image processing
Image processing is an important part of many scientific workflows
involving data collected in the form of images. Image processing can be
used to improve clarity in images, highlight specific content or objects,
extract quantitative information, and more. In the realm of quantitative
information extraction, sizes of objects and/or regions can be measured.

% x-radiography
X-radiography is the process of generating images using x-rays transmitted
through a material. Contrast in these images come from x-ray absorption
properties of the material being imaged, thus materials with localized
compositions and material structures/phases can be differentiated in these
images.

% computed tomography
X-ray computed tomography (XCT) is the process imaging a material in three
dimensions by capturing a series of two-dimensional x-radiographs from a
variety of angles around a central axis such that a three-dimensional
reconstruction of the images can be created. This reconstruction is
representative of the volume of the material imaged.

% additive manufacturing
Metal additive manufacturing (AM) is the process of creating a part
layer-by-layer, as opposed to more traditional subtractive processes.
AM is an umbrella term for a variety of processes that may include powder
or wire feedstock fused together in some way, typically by melting with
laser or electron beam.

% segmentation
Image segmentation is a form of image processing in which one or multiple
features in an image are separated from the rest of the image. This separation is done by splitting an image is split into
There are different methods for segmenting images, but the simplest is
by setting a threshold value, and creating a binary image
which defines a foreground of regions of the image that are of-interest,
and defines features as separate from the background.
This often requires image segmentation, in which an image is split into
two or more regions which can then be classified and analyzed to convey
information about intensity, size, location, shape, and/or distribution
of the features. Image segmentation can be performed on a single image,
or a volume (often represented as a series of images acting as slices).

