\subsection{Research Objectives and Questions}
The objective of this project was to explore the role of image processing
procedures in material investigation. This was done by investigating the
connection between x-radiography intensity and composition of a binary
alloy, developing procedures for tracking solid-liquid interfaces during
solidification experiments, extending watershed segmentation procedure to
improve results for nonuniform particles, and developing a software
package for informing physics simulation with initial conditions from
x-ray computed tomography data. The following questions were developed to
address these objectives:

\bigskip
\noindent 1. \textit{
    How can in-situ x-radiography be used in conjunction with
    other methods of analysis to infer composition of an Al-Ag alloy during
    solidification?
}

Studies have been performed using calibration experiments to relate x-ray
image intensity to composition during solidification of binary alloys. To
study whether these calibration experiments are necessary, directional
solidification of an Al-Ag alloy was monitored with in-situ, microfocus
radiography. The radiograph intensity was compared to energy dispersive
spectroscopy and the data were analyzed for similar trends.

\bigskip
\noindent 2. \textit{
    How successful is an image processing procedure at automating the
    identification, tracking, and velocity calculation of solid-liquid interfaces
    during in-situ solidification experiments?
}

Identifying, tracking, and analyzing the solid-liquid
interface in solidification experiments is necessary to extract important
information like solidification velocity and melt pool shape, but the task
of manually tracking interfaces is time consuming and prone to human error
and biases. Using traditional image processing methods, routines were
developed to identify, track, and analyze the solid-liquid interfaces in
additive manufacturing simulator experiments as well as rapid
solidification experiments captured with a dynamic transmission electron
microscope. The resulting melt pool locations were used to calculate
solidification velocities and compared to results from manually tracking
the interfaces.

\bigskip
\noindent 3. \textit{
    How can the segmentation of multi-sized, irregularly-shaped, and
    tightly-clustered particles be improved?
}

Watershed segmentation is a
common method for segmenting objects in an image from other objects and
from the background. One method of performing these types of algorithms is
by ``seeding'' the algorithm with ``marker'' points. Each marker will become a
segmented region, so controlling the marker points is one way to control
the results. Often markers are narrowed down using the distance between
adjacent points, but this distance can be extremely variable for
multi-sized and irregularly-shaped. Delaunay triangulation, which is an
algorithm which generates a connected grid of points, is tested to
determine if it can be used along with edge strength to filter markers
across varying distances.

\bigskip
\noindent 4. \textit{
    How can a workflow be designed to extract 3D geometries from x-ray
    computed tomography data such that the geometry of a physical sample can be
    reproduced digitally for use as initial conditions in an image-based physics
    simulation?
}

X-ray computed tomography
(XCT) is a useful tool for nondestructive testing of materials. A mock
high explosives system of F50 silica sand coated in Kel-F, a
polymeric binder, was subject to XCT to generate a set of 3D images.
\textit{Segmentflow}, a Python-based, segmentation workflow software
package was
developed to enable the segmentation of individual sand grains within the
XCT data. Based on parameters specified in an input file,
\textit{Segmentflow} can
output the segmented grains in a variety of formats to be used in physics
simulations.

